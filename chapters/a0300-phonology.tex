% !TEX root = ../main.tex
\chapter{Phonology}

As \textit{T'aalamuri}, Mazulia was known for its massive phonetic inventory, comparable to Quijada's Ithkuil, and for its lax phonotactics similar to that of Salish languages. Along its development, the language features a reduced set of sounds to sound more fluid, making it more natural. Mazulia’s phonological system consists of consonants, vowels, and stress as its core units.

\section{Consonants}
There are 27 consonant phonemes, each corresponds to multiple allophones, resulting in 95 to 125 different phones depending on dialect. The phonemes are given in Table \ref{tab:3-consonant-phonemes}.

\begin{table}[h]
\caption{Consonant Phonemes}
\label{tab:3-consonant-phonemes}
\begin{tabular}{|r|cc|cc|cc|cc|c|cc|c|}
\hline
\multicolumn{1}{|c|}{\textbf{}} & \multicolumn{2}{c|}{\textbf{Labial}}           & \multicolumn{2}{c|}{\textbf{Dental}}           & \multicolumn{2}{c|}{\textbf{Coronal}}          & \multicolumn{2}{c|}{\textbf{Laminal}}          & \textbf{Palatal} & \multicolumn{2}{c|}{\textbf{Dorsal}}           & \textbf{Glottal} \\ \cline{2-13} 
\multicolumn{1}{|c|}{\textbf{}} & \multicolumn{1}{c|}{\textit{-v}} & \textit{+v} & \multicolumn{1}{c|}{\textit{-v}} & \textit{+v} & \multicolumn{1}{c|}{\textit{-v}} & \textit{+v} & \multicolumn{1}{c|}{\textit{-v}} & \textit{+v} & \textit{+v}      & \multicolumn{1}{c|}{\textit{-v}} & \textit{+v} & \textit{-v}      \\ \hline
\textbf{Occlusive}              & \multicolumn{1}{c|}{p}           & b           & \multicolumn{1}{c|}{t}           & d           & \multicolumn{1}{c|}{ts}          & ɾ           & \multicolumn{1}{c|}{t͡ʃ}         &             &                  & \multicolumn{1}{c|}{k, q}        & g           & ʔ                \\ \hline
\textbf{Nasal}                  & \multicolumn{1}{c|}{}            & m           & \multicolumn{1}{c|}{}            &             & \multicolumn{1}{c|}{}            & n           & \multicolumn{1}{c|}{}            &             & ɲ                & \multicolumn{1}{c|}{}            & ŋ           &                  \\ \hline
\textbf{Fricative}              & \multicolumn{1}{c|}{f}           & v           & \multicolumn{1}{c|}{ɬ}           &             & \multicolumn{1}{c|}{s}           & z           & \multicolumn{1}{c|}{ʃ}           & ɹ           &                  & \multicolumn{1}{c|}{x}           & ɣ           & h                \\ \hline
\textbf{Liquid}            & \multicolumn{1}{c|}{}            & w           & \multicolumn{1}{c|}{}            & l           & \multicolumn{1}{c|}{}            &             & \multicolumn{1}{c|}{}            &             & j                & \multicolumn{1}{c|}{}            &             &                  \\ \hline
\end{tabular}
\end{table}

Being a universal language akin to English, every phoneme in Mazulia has multitudes of sometimes seemingly unrelated allophones. In most cases, the allophones follow the phonetic environment of the surrounding sounds, for example, between vowels, in the final position, etc. The following subsections describe Mazulia's consonant allophones.

Note that, unlike Slavic or Caucasic languages, palatalized and labialized consonants are not counted as separate phonemes.

\subsection{Labials}

\subsubsection{Unvoiced bilabial plosive /p/}
The sound /p/ defaults to unaspirated unvoiced bilabial plosive [p] like in the following words:

\begin{tabular}{rllll}
initial         & {[}p{]}  & \textit{paidā-}     & {[}pa̠iˈðɛ̞{]}        & 'benefit'      \\
intervocalic    & {[}p{]}  & \textit{ȳpia-}      & {[}ˈiᶻ.pi.ʌ{]}        & 'to freeze'    \\
geminated       & {[}p.p{]}  & \textit{beppu}      & {[}ˈbɛp.pʉ{]}         & 'movie'        \\
cluster initial & {[}p{]}  & \textit{prexmaitaš} & {[}ˈpɾɛ.xmɐi̯.t̪ɐs̠{]} & 'introduction' \\
final           & {[}p̚{]} & \textit{qop-}       & {[}qʰɒp̚{]}           & 'to stop'     
\end{tabular}


Rare aspirated allophone [pʰ] typically occurs when /p/ precedes /h/, mostly in loanwords, or sometimes before another consonant.


\begin{tabular}{lll}
\textit{qopna}  & {[}qɒpʰˈna̠{]}   & 'stopping' \\
\textit{īmphal} & {[}jimˈpʰæ̠l̠{]} & 'head'    
\end{tabular}


Palatalized [pʲʰ] normally occurs before palatalizing vowels ⟨ē⟩, ⟨i⟩, ⟨ī⟩, and ⟨ü⟩ or /j/ + vowel, but this is not always the case. Its unaspirated version [pʲ] occurs at the end of a word ending in ⟨-py⟩ /pj/).


\begin{tabular}{lll}
\textit{pēctā} & {[}pʲʰɘt̪͡s̪ˈt̪ʰæ{]} & 'post', 'postal'          \\
\textit{fipy}  & {[}fʲiᶻpʲ{]}          & 'toddler', 'small animalᶻ
\end{tabular}

\subsubsection{Voiced bilabial plosive /b/}
The sound /b/ has the following allophones: [b], [β]~[w], [ʋ], [bʲ], and [βʲ]. The default sound is [b] at all positions, except for codas where some Central dialects pronounce it slightly buccalized [b̫]:

\begin{tabular}{rllll}
initial         & {[}b{]}   & \textit{byk}        & {[}bɯk̚{]}          & 'passenger bus' \\
intervocalic    & {[}b{]}   & \textit{barēbar-}   & {[}bɑɽˈɛ.bɑɽ{]}     & 'to sustain'    \\
geminated       & {[}b.b{]} & \textit{ībban-}     & {[}ibˈba̠n{]}       & 'to support'    \\
cluster initial & {[}b{]}   & \textit{bžedat-}    & {[}bz̠eˈða̠t̪̚{]}   & 'to predate'    \\
after nasal     & {[}b{]}   & \textit{bambaŋ}     & {[}ba̠mˈba̠ŋ{]}     & 'rope'          \\
final           & {[}b̫{]}   & \textit{fet-pradob} & {[}fet̪.pɾʌˈdɒb̫{]} & 'tyranny'      
\end{tabular}

Voiced bilabial fricative [β] to voiced labial–velar approximant [w] usually occur (1) in an unstressed intervocalic environment, (2) in non-intial position in a cluster, (3) in a medial coda before a non-bilabial plosive or nasal consonant, or (4) in final position after a sequence of closed back vowels. They are interchangeable, but some speakers have their own set of preferences on which to use for which word.

\begin{tabular}{rlll}
case (1)                     & \textit{teber}      & {[}ˈt̪ɘ.wɘɾ{]}          & 'plate'             \\
case (2)                     & \textit{hyilachba-} & {[}ˈhɨi̯.l̪ˠɑ.t͡s̠βɑ{]} & 'to save money'     \\
case (2)                     & \textit{rëšbod}     & {[}ɾɨˈs̠βɒd{]}          & 'angel (male)' \\
case (3)                     & \textit{tëbt}       & {[}t̪ɨwt̪{]}            & 'apartment'         \\
case (4) & \textit{gūrëb}               & {[}ˈɡuᵝ.ɾɞβ{]}          & 'leg'               \\
case (4) & \textit{tsyb}                & {[}t̪s̪ɯw{]}            & 'edge'             
\end{tabular}

Voiced labiodental approximant [ʋ] only occurs after an aspirated stop.
\begin{tabular}{rll}
\textit{tba}      & {[}t̪ʰʋɑ{]}          & 'reservoir'             \\
\textit{čbēr}      & {[}cʰʋɘɾ{]}          & 'snail'             \\
\end{tabular}

Palatalized [bʲ] and [βʲ] occur when ⟨ē⟩, or /i/~/j/ + vowel follows the unpalatalized counterpart.

\begin{tabular}{lll}
\textit{bēt} & {[}bʲe̝t̪̚{]} & 'sea'          \\
\textit{ǧobi-}  & {[}ʁœβʲ{]}          & 'to threaten'
\end{tabular}

\subsubsection{Voiced bilabial nasal /m/}
This sound is consistently voiced bilabial nasal [m] in non-palatalized environment, thus has only one allophone [mʲ]. However, [m] can also become vocalic [m̩] when a consonant cluster is too crowded.

\begin{tabular}{rlll}
initial & \textit{mūn}     & {[}muᵝn{]}          & 'size'  \\
medial  & \textit{gemel}   & {[}ˈɡʲɛ.mɛl̠{]}     & 'camel' \\
final   & \textit{paschīm} & {[}pa̠s̪.t͡s̠i̟m{]} & 'peace' \\
vocalic & \textit{mbrai}   & {[}m̩ˈbɾa̠i̯{]}     & 'bay'  
\end{tabular}

Palatalization of [m] into [mʲ] happens before (1) palatalizing vowel ⟨ē⟩, (2) combination of ⟨mi⟩ + vowel, or (3) before the consonant /j/.

\begin{tabular}{rlll}
case 1 & \textit{mēgre-}   & {[}ˈmʲi.ɡɾæ{]}     & 'heavy' (measurement)                 \\
case 2 & \textit{miei}     & {[}mʲɛe̯{]}     & 'Miss.'      \\
case 3 & \textit{mīmy}  & {[}miᶻmʲ{]}      & 'meme' \\
\end{tabular}

In some cases, /m/ coda nasalizes and/or closes the preceding vowel or diphthong.

\begin{tabular}{rll}
\textit{ctum-}   & {[}t̪͡s̪̩ˈtʊ̃(m){]}     & 'to pop'\\
\textit{pem-}     & {[}pɛ̃m{]}     & 'distorted'      \\
\textit{sei·īm}  & {[}sei̯ˈʔiʊ̯m{]}      & 'drought' \\
\textit{qalym}  & {[}ˈqɑ.lˠɜ̃ũ̯m{]}      & 'ear' \\
\end{tabular}


\subsubsection{Unvoiced labiodental fricative /f/}
The sound /f/ has a unique cultural background. The variations unvoiced labiodental fricative [f] and unvoiced bilabial fricative [ɸ] do not depend on the phonetic environment, rather, whether it is used within a pejorative words or a neutral word: ⟨f⟩ [f] for negative-sounding words, else ⟨hv⟩ [ɸ]. Both variants can be palatalized before some ⟨i⟩, ⟨ī⟩, and ⟨y⟩ /j/.

\begin{tabular}{rllll}
initial, pejorative     & {[}f{]}  & \textit{fafal}   & {[}fa̠ˈfa̠l{]}    & 'trouble'                 \\
medial, pejorative      & {[}f{]}  & \textit{beaftař} & {[}bæ.a̠fˈt̪ɑɽ{]} & 'hell'                    \\
final, pejorative       & {[}f{]}  & \textit{ǧožof}   & {[}ʁɒˈz̠ɒf{]}     & 'measles'                 \\
palatalized, pejorative & {[}fʲ{]} & \textit{fipy}    & {[}fʲiᶻpʲ{]}      & 'toddler', 'small animal' \\
initial, neutral        & {[}ɸ{]}  & \textit{hvulsu}  & {[}ˈɸʊ̟l̠.s̪ʊ̟{]} & 'capital' (money)         \\
medial, neutral         & {[}ɸ{]}  & \textit{xohvi}            & {[}ˈxʌ.ɸɪ{]}      & 'coffee'                  \\
final, neutral          & {[}ɸ{]}  & \textit{këhv-}            & {[}kɜɸ{]}         & 'to suggest'              \\
palatalized, neutral    & {[}ɸʲ{]} & \textit{hvil-}            & {[}ɸʲɪlʲ{]}        & 'to fold'                
\end{tabular}

\subsubsection{Voiced labiodental fricative /v/}
The /v/ phoneme has four allophones: [v], [ʋ], [w], and [vʲ], depending on the environment.

The voiced labiodental fricative [v] appears (1) in the initial position of stressed syllable, (2) in the final position after ⟨ō⟩, ⟨ö⟩, ⟨ū⟩, and ⟨ü⟩, (3) as the first component of a cluster, (4) and in the very first letter of a word.

\begin{tabular}{rlll}
case 1     & \textit{vüv-}    & {[}vʏv{]}    & 'wet'                 \\
case 2     & \textit{tyva}    & {[}t̪ʰəˈvɑ{]} & 'precious', 'glorious'                    \\
case 3     & \textit{vlica}   & {[}vlɪˈt̪͡s̪a̠{]}     & 'street', 'small road'                 \\
case 4     & \textit{vatu}    & {[}vaˈt̠ʊ{]}      & 'purpose' \\
\end{tabular}

Voiced labiodental approximant [ʋ] occurs in the non-initial position of a cluster and between vowels in an unstressed syllable.

\begin{tabular}{rlll}
cluster non-initial      & \textit{kvach-}   & {[}kʰʋa̠t͡s̠{]}     & 'to endure'                 \\
intervocalic, unstressed & \textit{arvūne-}  & {[}aɾ.ʋüˈnɛ{]}      & 'to enable (electronics)' \\
\end{tabular}

Syllable-finally except after ⟨ō⟩, ⟨ö⟩, ⟨ū⟩, and ⟨ü⟩, /v/ is pronounced as voiced labial–velar approximant [w] or less commonly voiced labiodental approximant [ʋ].

\begin{tabular}{rlll}
{[}w{]} & \textit{tavdom}   & {[}t̪ʌw͜ˈd̪ɒ̃m{]}     & 'debt'                 \\
{[}w{]} & \textit{alëv}     & {[}ˈa̠.lɜw{]}     & 'life'      \\
{[}ʋ{]} & \textit{tīmniv}  & {[}ˈt̪ʲʰiᶻm.n̪ɪʋ{]}      & 'promise' \\
{[}ʋ{]} & \textit{sadlēv}  & {[}s̪a̠dˈlʲɪə̯ʋ{]}      & 'republic' \\
\end{tabular}

Palatalized [vʲ] happens normally before palatalizing vowels ⟨ē⟩, ⟨i⟩, ⟨ī⟩, or the consonant /j/.

\begin{tabular}{rll}
\textit{vēnda-}   & {[}vʲɘn̪ˈd̪ʌ{]}     & 'to comfort', 'to console'                 \\
\textit{-vīk}     & {[}vʲiᶻcʰ{]}     & \textsc{nz}      \\
\textit{uvyi}  & {[}ʊˈvʲɪ{]}      & 'belonging' \\
\end{tabular}

\subsubsection{Voiced Labial–velar approximant /w/}
The voiced labial–velar approximant /w/ phoneme has three allophones: [w], [ʋ] and the palatalized variant [ɥ]. 
The phone [w] appears (1) at the initial and coda positions of a syllable and (2) in non-initial position of clusters.

\begin{tabular}{rlll}
case 1     & \textit{wēlu}    & {[}ˈwe.lu{]}    & 'ending'                 \\
case 1     & \textit{sëgaw}    & {[}s̪ɜˈɣa̠w{]} & 'infidel'                    \\
case 2     & \textit{twen-}   & {[}t̪wen{]}     & 'grand'                 \\
case 2     & \textit{xwat-}    & {[}xwa̠t̪{]} & 'to believe in something' \\
\end{tabular}

The voiced labiodental approximant allophone [ʋ] occurs in (1) loanwords that sound German or Polish and (2) in the initial position of clusters.

\begin{tabular}{rlll}
case 1     & \textit{walega}    & {[}ʋa̠ˈlɛ.ɡa̠{]}    & 'rebate'                 \\
case 2     & \textit{wrica-}   & {[}ʋɾɪˈt̪͡s̪a̠{]}     & 'to interact with joy'                 \\
\end{tabular}

When palatalized, /w/ becomes voiced labial-palatal approximant [ɥ]. There is no strict rule that indicates if /w/ is palatalized or not. Generally, when /w/ is adjacent to ⟨ī⟩, ⟨ü⟩, or consonant ⟨y⟩, it is palatalized. Some examples are:

\begin{tabular}{lll}
\textit{wilū-}   & {[}ˈɥi.luᵝ{]}     & 'sadness', 'lament'                 \\
\textit{qurwyün}     & {[}qɔɾˈɥʏn̪{]}     & 'mosque'      \\
\textit{mawa}  & {[}ˈma̠.ɥa̠{]}      & 'city (\textsc{/ra/e}: East Rom Atasi dialect)' \\
\end{tabular}

\subsection{Dentals}
\subsubsection{Unvoiced dental plosive /t/}
The unvoiced dental plosive /t/ is generally dental [t̪] in any position. Historical /ʈ/ sound and /t/ after laminals or [l̠] become alveolar [t].
As a non-final component of a cluster except before rhotics, sibilants, and /j/ 
or as a final component of a cluster coda, the sound is more often aspirated ([t̪ʰ], [tʰ]).

\begin{tabular}{rlll}
dental, unaspirated & \textit{tempeř-}   & {[}ˈt̪ɛ̃m.pɛr̝{]}     & 'double'                 \\
dental, unaspirated & \textit{hcot}     & {[}həˈt̪͡s̪ɒt̪{]}     & 'dog'      \\
alveolar, unaspirated & \textit{čelt}  & {[}t̠͡s̠ɛ̞l̠t{]}      & 'frigid' \\
alveolar, unaspirated & \textit{tūvin-}  & {[}ˈtu.βɪn{]}      & 'connecting' \\
dental, aspirated & \textit{ovamt-}   & {[}ɔʋˈamt̪ʰ{]}     & 'to be larger than (math)'                 \\
dental, aspirated & \textit{atxna}     & {[}ˈa̠t̪ʰ.xna̠{]}     & 'mountain'      \\
alveolar, aspirated & \textit{twen-}  & {[}tʰwɛn{]}      & 'frigid' \\
\end{tabular}

When palatalized, it is always an aspirated dental [t̪ʲʰ] in non-final positions and [t̪ʲ] finally. 
Labialization can only happen at coda and it is always dental and unaspirated ([t̪ʷ]). 
Simultaneously palatalized and labialized /t/ only occurs initially as an aspirated [t̪ᶣʰ].

\begin{tabular}{rlll}
palatalized, unaspirated & \textit{tuviřēti}   & {[}t̪ʊ.ʋɪˈr̝ʲe̝t̪ʲ{]}     & 'existence' \\
palatalized, aspirated & \textit{tēn}   & {[}t̪ʲʰɪᶻn{]}     & 'day' \\
labialized & \textit{xatw}   & {[}χ̠a̠t̪ʷ{]}     & 'navy' \\
palatalized, labialized & \textit{tüir-}   & {[}t̪ᶣiᶻr̝{]}     & 'old (person)' \\
\end{tabular}

\subsubsection{Voiced dental plosive /d/}
The voiced dental plosive /d/ is generally alveolar [d] in any position. 
Historical /ð/ phoneme and final cluster component after dentals, sibilants, rhotics, /v/, or /ɲ/ become dental [d̪].

\begin{tabular}{rlll}
alveolar & \textit{dāpe}   & {[}ˈdæ.pe̝{]}     & 'chicken' \\
alveolar & \textit{ētsed}   & {[}jɘt̪ˈs̪ɛd{]}     & 'existence' \\
dental & \textit{suvdeiŋ-}   & {[}sʊwˈd̪e̞i̯ŋ{]}     & 'attractive (personality)' \\
\end{tabular}

Lenition of /d/ into voiced dental fricative [ð] happens in intervocalic settings or after liquids.
In some examples between ⟨ë⟩s or /ɨ/ or a final code after aforementioned sounds, 
it is like Danish \textit{soft d}: retracted velarized dental approximant [ð̞˗ˠ].

\begin{tabular}{rlll}
plain & \textit{tadër-}   & {[}täˈðəɾ{]}     & 'to sleep' \\
plain & \textit{vrdō}   & {[}ˈvr̩.ðɞ{]}     & 'platform (physical)' \\
velarized & \textit{gōvëdë}   & {[}ˈɡɵᵝ.vɤ.ð̞˗ˠɤ{]}     & 'existence' \\
velarized & \textit{žëd-}   & {[}z̠ɤð̞˗ˠ{]}     & 'sad' \\
\end{tabular}

\subsubsection{Unvoiced alveolar lateral fricative /ɬ/}
